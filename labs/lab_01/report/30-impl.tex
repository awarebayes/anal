\chapter{Технологическая часть}

В данном разделе будут приведены требования к программному обеспечению, средства реализации и листинги кода.

\section{Требования к ПО}

К программе предъявляется ряд требований:
\begin{itemize}
	\item у пользователя есть выбор алгоритма, или какой-то один, или все сразу, а также есть выбор тестирования времени;
	\item на вход подаются две строки на русском или английском языке в любом регистре;
	\item на выходе — искомое расстояние для выбранного метода (выбранных методов) и матрицы расстояний для матричных реализаций.
\end{itemize}

\section{Средства реализации}

В качестве языка программирования для реализации данной лабораторной работы был выбран ЯП Python \cite{pythonlang}. 

Данный язык достаточно удобен и гибок в использовании. 

Время работы алгоритмов было замерено с помощью функции process\_time() из библиотеки time \cite{pythonlangtime}

\section{Сведения о модулях программы}
Программа состоит из двух модулей:
\begin{enumerate}
	\item main.py - главный файл программы, в котором располагаются коды всех алгоритмов и меню;
	\item test.py - файл с замерами времени для графического изображения результата.
\end{enumerate}


\section{Листинг кода}

 В листингах \ref{lst:lev_rec}, \ref{lst:lev_mat}, \ref{lst:lev_rec_mat}, \ref{lst:dlev} приведены реализации алгоритмов нахождения расстояния Левенштейна и Дамерау--Левенштейна.

\begin{lstlisting}[label=lst:lev_rec,caption=Функция нахождения расстояния Левенштейна с использованием рекурсии.]
def levenshtein_recursive(str1, str2, out_put = False):
	n = len(str1)
	m = len(str2)
	
	if n == 0 or m == 0:
		return abs(n - m)
	
	flag = 0
	if str1[-1] != str2[-1]:
		flag = 1
	
	min_lev_rec = min(levenshtein_recursive(str1[:-1], str2) + 1,
					  levenshtein_recursive(str1, str2[:-1]) + 1,
					  levenshtein_recursive(str1[:-1], str2[:-1]) + flag)
	
	return min_lev_rec
	
\end{lstlisting}

\begin{lstlisting}[label=lst:lev_mat,caption=Функция нахождения расстояния Левенштейна с использованием рекурсии.]
def levenshtein_matrix(str1, str2, out_put = True):
	n = len(str1)
	m = len(str2)
	
	matrix = create_matrix(n + 1, m + 1)
	
	for i in range(1, n + 1):
		for j in range(1, m + 1):
			add, delete, change = matrix[i - 1][j] + 1,\
								  matrix[i][j - 1] + 1,\
						    	  matrix[i - 1][j - 1]
			if str2[j - 1] != str1[i - 1]:
				change += 1
			else:
				change += 0
			
			matrix[i][j] = min(add, delete, change)
	
	if out_put:
		print_matrix(str1, str2, matrix)
	
	return matrix[n][m]
	
\end{lstlisting}

\begin{lstlisting}[label=lst:lev_rec_mat,caption=Функция нахождения расстояния Левенштейна с использованием рекурсии.]
def levenshtein_matrix_recursive(str1, str2, out_put = True):
	n = len(str1)
	m = len(str2)
	
	def recursive(str1, str2, n, m, matrix):
		if (matrix[n][m] != -1):
			return matrix[n][m]
		
		if (n == 0):
			matrix[n][m] = m
			return matrix[n][m]
		
		if (n > 0 and m == 0):
			matrix[n][m] = n
			return matrix[n][m]
		
		delete = recursive(str1, str2, n - 1, m, matrix) + 1
		add = recursive(str1, str2, n, m - 1, matrix) + 1
		
		flag = 0
		
		if (str1[n - 1] != str2[m - 1]):
			flag = 1
		
		change = recursive(str1, str2, n - 1, m - 1, matrix) + flag
		
		matrix[n][m] = min(add, delete, change)
		
		return matrix[n][m]
	
	matrix =  create_matrix(n + 1, m + 1)
	
	for i in range(n + 1):
		for j in range(m + 1):
			matrix[i][j] = -1
	
	recursive(str1, str2, n, m, matrix)
	
	if out_put:
		print_matrix(str1, str2, matrix)
	
	return matrix[n][m]
	
\end{lstlisting}

\begin{lstlisting}[label=lst:dlev,caption=Функция нахождения расстояния Левенштейна с использованием рекурсии.]
def damerau_levenshtein_recursive(str1, str2, out_put = False):
	n = len(str1)
	m = len(str2)
	
	if n == 0 or m == 0:
		if n != 0:
			return n
		if m != 0:
			return m
		return 0
	
	change = 0
	if str1[-1] != str2[-1]:
		change += 1
	
	if n > 1 and m > 1 and str1[-1] == str2[-2] \
		and str1[-2] == str2[-1]:
		min_ret = min(damerau_levenshtein_recursive(str1[:n - 1], str2) + 1,
					  damerau_levenshtein_recursive(str1, str2[:m - 1]) + 1,
				      damerau_levenshtein_recursive(str1[:n - 1], str2[:m - 1]) + change,
		              damerau_levenshtein_recursive(str1[:n - 2], str2[:m - 2]) + 1)
	else:
		min_ret = min(damerau_levenshtein_recursive(str1[:n - 1], str2) + 1,
		              damerau_levenshtein_recursive(str1, str2[:m - 1]) + 1,
		              damerau_levenshtein_recursive(str1[:n - 1], str2[:m - 1]) + change)
	return min_ret
	
\end{lstlisting}

\section{Функциональные тесты}
В таблице \ref{tabular:functional_test} приведены функциональные тесты для алгоритмов вычисления расстояния Левенштейна (в таблице столбец подписан "Левенштейн") и Дамерау — Левенштейна (в таблице - "Дамерау-Л."). Все тесты пройдены успешно.


\begin{table}[h]
	\begin{center}
		\caption{\label{tabular:functional_test} Функциональные тесты}
		\begin{tabular}{|c|c|c|c|c|}
			\hline
			& & & \multicolumn{2}{c|}{Ожидаемый результат} \\
			\hline
			№&Строка 1&Строка 2&Левенштейн&Дамерау-Л. \\
			\hline
			1&скат&кот&2&2 \\
			\hline
			2&машина&малина&1&1 \\
			\hline
			3&дворик&доврик&2&1 \\
			\hline
			4&"пустая строка"&университет&11&11 \\
			\hline
			5&сентябрь&"пустая строка"&8&8 \\
			\hline
			8&тело&телодвижение&8&8 \\
			\hline
			9&ноутбук&планшет&7&7 \\
			\hline
			10&глина&малина&2&2 \\
			\hline
			11&рекурсия&ркерусия&3&2 \\
			\hline
			12&браузер&баурзер&2&2 \\
			\hline
			13&bring&brought&4&4 \\
			\hline
			14&moment&minute&4&4 \\ 
			\hline
			15&person&eye&5&5 \\
			\hline
			16&week&weekend&3&3 \\
			\hline 
			17&city&town&4&4 \\
			\hline
		\end{tabular}
	\end{center}
\end{table}


\section*{Вывод}

Были разработаны и протестированы алгоритмы: нахождения расстояния Левенштейна рекурсивно, с заполнением матрицы и рекурсивно с заполнением матрицы, а также нахождения расстояния Дамерау — Левенштейна рекурсивно.
