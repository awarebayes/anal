\chapter{Технологический раздел}
В этом разделе будут приведены листинги кода и результаты функционального тестирования.
Также будет произведена оценка трудоёмкости алгоритмов.

\section{Средства реализации программного обеспечения}
В качестве языка для описания алгоритмов был выбран Python \cite{10.5555/1593511}.
Алгоритмы реализованы с помощью библиотеки для JIT компиляции кода Numba \cite{numba} и совместимой с ней библиотекой линейной алгебры numpy.
Numba позволяет транспилировать код Python в IR-код, который после компилируется в байткод LLVM, который затем может быть исполнен на CPU или GPU.
Для замера процессорного времени используется функция process\_time\_ns из библиотеки time. 
В её результат не включается время, когда процессор не выполняет задачу \cite{python}. 
Оптимизация скомпилированного кода была отключена путем установки переменной окружения NUMBA\_OPT=0.

\section{Реализация алгоритмов}
\FloatBarrier
На листингах 3.1, 3.2 и 3.3 приведены реализации различных алгоритмов умножения матриц.
\begin{lstinputlisting}[language=Python, caption=Реализация алгоритма классического умножения матриц, linerange={13-26},
	basicstyle=\small\sffamily, frame=single]{src/algs.py}
\end{lstinputlisting}
\FloatBarrier

\FloatBarrier
\begin{lstinputlisting}[language=Python, caption=Реализация алгоритма Винограда, linerange={28-67},
	basicstyle=\small\sffamily, frame=single]{src/algs.py}
\end{lstinputlisting}
\FloatBarrier

\FloatBarrier
\begin{lstinputlisting}[language=Python, caption=Реализация оптимизированного алгоритма Винограда, linerange={69-105},
	basicstyle=\small\sffamily, frame=single]{src/algs.py}
\end{lstinputlisting}
\FloatBarrier

\section{Функциональные тесты}
Используется метод черного ящика.
В таблице \ref{tabular:functional_test} приведены результаты функциональных тестов. 
В первом столбце -- первый операнд умножения матриц.
Во втором столбце -- второй операнд умножения матриц.
В третьем столбце -- ожидаемый результат.

\begin{table}[h!]
	\centering
	\begin{center}
		\begin{threeparttable}
		\caption{Функциональные тесты}
		\label{tabular:functional_test}
		\begin{tabular}{| c | c | c |}
			\hline
			Матрица 1 & Матрица 2 & Ожидаемый результат \\ \hline
			\begin{tabular}{c c c} 
				1 & 2 & 3 \\
				4 & 5 & 6 \\
				7 & 8 & 9 \\
			\end{tabular}
			&
			\begin{tabular}{c c c} 
				9 & 8 & 7 \\
				6 & 5 & 4 \\
				9 & 8 & 7 \\
			\end{tabular}
			&
			\begin{tabular}{c c c} 
				30 & 24 & 18 \\
				84 & 69 & 54 \\
				138 & 114 & 90 \\
			\end{tabular}\\
			\hline
			\begin{tabular}{c c c c} 
				1 & 2 & 3 & 4\\
				5 & 6 & 7 & 8 \\
				9 & 10 & 11 & 12 \\
				13 & 14 & 15 & 16
			\end{tabular}
			&
			\begin{tabular}{c c c c} 
				16 & 15 & 14 & 13 \\
				12 & 11 & 10 & 9 \\
				8 & 7 & 6 & 5 \\
				4 & 3 & 2 & 1 \\
			\end{tabular}
			&
			\begin{tabular}{c c c c} 
				80 & 70 & 60 & 50 \\
				240 & 214 & 188 & 162 \\
				400 & 358 & 316 & 274 \\
				560 & 502 & 444 & 386 \\
			\end{tabular}\\
			\hline
			\begin{tabular}{c c c} 
				1 & 2 & 3 \\
				4 & 5 & 6 \\
			\end{tabular}
			&
			\begin{tabular}{c c} 
				6 & 5 \\
				4 & 3 \\
				2 & 1 \\
			\end{tabular}
			&
			\begin{tabular}{c c} 
				20 & 14 \\
				56 & 41 \\
			\end{tabular} \\
			\hline
			\begin{tabular}{c c c c} 
				1 & 0 & 0 & 0\\
				0 & 1 & 0 & 0 \\
				0 & 0 & 1 & 0 \\
				0 & 0 & 0 & 1 \\
			\end{tabular}
			&
			\begin{tabular}{c c c c} 
				1 & 2 & 3 & 4 \\
				5 & 6 & 7 & 8 \\
				9 & 10 & 11 & 12 \\
				13 & 14 & 15 & 16 \\
			\end{tabular}
			&
			\begin{tabular}{c c c c} 
				1 & 2 & 3 & 4 \\
				5 & 6 & 7 & 8 \\
				9 & 10 & 11 & 12 \\
				13 & 14 & 15 & 16 \\
			\end{tabular}\\
			\hline
			\begin{tabular}{c c c c} 
				0 & 0 & 0 & 0\\
				0 & 0 & 0 & 0 \\
				0 & 0 & 0 & 0 \\
				0 & 0 & 0 & 0 \\
			\end{tabular}
			&
			\begin{tabular}{c c c c} 
				1 & 2 & 3 & 4 \\
				5 & 6 & 7 & 8 \\
				9 & 10 & 11 & 12 \\
				13 & 14 & 15 & 16 \\
			\end{tabular}
			&
			\begin{tabular}{c c c c} 
			0 & 0 & 0 & 0\\
			0 & 0 & 0 & 0 \\
			0 & 0 & 0 & 0 \\
			0 & 0 & 0 & 0 \\
			\end{tabular}\\
			\hline
			\begin{tabular}{c c c} 
				1 & 2 & 3 \\
				4 & 5 & 6 \\
			\end{tabular}
			&
			\begin{tabular}{c c c} 
				9 & 8 & 7 \\
				6 & 5 & 4\\
			\end{tabular}
			&
			- \\
			\hline
			\begin{tabular}{c c c} 
				1 & 2 & 3 \\
				4 & 5 & 6 \\
			\end{tabular} 
			&
			* 
			&
			- \\
			\hline
		\end{tabular}
	\end{threeparttable}
	\end{center}
\end{table}

\pagebreak

* -- пустая матрица.

- -- данные введены некорректно.

Все функциональные тесты были отработаны успешно.

\section{Модель для расчёта трудоёмкости алгоритмов}
Введем модель трудоемкости для оценки алгоритмов:
\begin{itemize}
	\item для базовых операции: +, -, *, /, =, ==, <=, >=, !=, +=, [], ++ -- трудоёмкость равна 1;
	\item трудоёмкость условия \textit{if УСЛОВИЕ then A else B} будет подсчитана по формуле \ref{eq}:
	\begin{equation}
		\label{eq}
		F = F_{условия} +
		\begin{cases}
			F_A &\text{, если условие выполняется}\\
			F_B &\text{, иначе}
		\end{cases}
	\end{equation}.
	\item Трудоймкость цикла \textbf{for} будет подсчитана по формуле \ref{eq:1}:
	\begin{equation}
		\label{eq:1}
		F_{инициализации} + F_{сравнения} + 
		N(F_{тела} + F_{инициализации} + F_{сравнения})
	\end{equation}
	\item Трудоёмкость вызова функции равна 0.
\end{itemize}

\section{Вычисление трудоёмкости для алгоритмов}
Для каждого из алгоритмов проведём вычисление трудоёмкости.
Для всех случаев размеры первой матрицы -- $n_1 \times m_1$, а второй -- $n_2 \times m_2$.

\subsection{Вычисление трудоёмкости классического алгоритма}
Инициализация матрицы результата имеет следующую Трудоемкость:
\begin{equation}
	f_{init} = 1 + 1 + n_1(1 + 2 + 1) + 1 = 4n_1 + 3
\end{equation}

Трудоемкость классического алгоритма вычисляется как:
\begin{equation}
f_{class} = 1 + n_1(1 + (1 + m_2(1 + (1 + m_1(1 + (8) + 1) + 1) + 1) + 1) + 1) + 1
\end{equation}

\begin{equation}
f_{class} = n_1(m_2(10m_1 + 4) + 4) + 4) + 2 = 10n_1m_2m_1+ 4n_1m_2 + 4n_1 +2
\end{equation}

\subsection{Вычисление трудоёмкости неоптимизированного алгоритма Винограда}
Цикл №1 имеет трудоёмкость:
\begin{equation}
	f_{c1} = \frac{15}{2}n_1m_1 + 5n_1 + 2
\end{equation}

Цикл №2 имеет следующую трудоёмкость: 
\begin{equation}
	f_{c2} = \frac{15}{2}n_2m_2 + 5n_2 + 2
\end{equation}

Цикл №3 имеет следующую трудоемкость:
\begin{equation}
	f_{c3} = 13n_1m_2m_1 + 12n_1m_2 + 4n_1 + 2
\end{equation} 

Условный переход имеет следующую трудоемкость: 
\begin{equation}
	f_{if} = \begin{cases}
	2    &\text{условие не выполняется}\\
	15n_1m_2 + 4n_1 + 2 &\text{условие выполняется}\\
	\end{cases}
\end{equation}

Итоговая трудоемкость:

\begin{align*} 
	13n_1m_2m_1 + \frac{15}{2}n_1m_1 +\frac{15}{2}m_2n_2 + 12n_1m_2 + 5n_1 + \\
	+ 5m_2 + 4n_1 + 6 + 
	\begin{cases}
	2    &\text{условие не выполняется}\\
	15n_1m_2 + 4n_1 + 2 &\text{условие выполняется}\\
	\end{cases}
\end{align*}

\subsection{Вычисление трудоёмкости оптимизированного алгоритма Винограда}
Цикл №1 имеет трудоемкости: 

\begin{equation}
\frac{11}{2}n_1m_1 + 4n_1 + 2 
\end{equation}

Трудоемкость для цикла №2:
\begin{equation}
\frac{11}{2}m_2n_2+ 4m_2 + 2
\end{equation}

Трудоемкость для цикла №3:
\begin{equation}
	\frac{17}{2}n_1m_2m_1 + 9n_1m_2 + 4n_1 + 2
\end{equation}



Условный переход имеет трудоемкость:
\begin{equation}
\begin{cases}
	1    &\text{условие не выполняется}\\
	10n_1m_2 + 4n_1 + 2 &\text{условие выполняется}\\
\end{cases}
\end{equation}

Итоговая трудоемкость:
\begin{align*}
 \frac{17}{2}n_1m_2m_1 + \frac{11}{2}n_1m_1 + \frac{11}{2}m_2n_2 + 9n_1m_2 \\
  + 8n_1 + 4m_2 + 6 + \begin{cases}
	1    &\text{условие не выполняется}\\
	10n_1m_2 + 4n_1 + 2 &\text{условие выполняется}\\
  \end{cases}
\end{align*}

\section*{Вывод}

Сформированы требования, предъявляемые к ПО, приведены листинги кода.
Были вычислены трудоёмкости алгоритмов, написаны и запущены функциональные тесты.
Все реализации алгоритма успешно прошли тесты.
Для реализаций были вычисены трудоемкости.