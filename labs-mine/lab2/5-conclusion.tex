\chapter{Заключение}

В ходе выполения лабораторной работы были выполнены следуюзие поставленные задачи.

\begin{itemize}
	\item рассмотрены существующие алгоритмы умножения матриц;
	\item приведена схемы реализации рассматриваемых алгоритмов;
	\item определены средства программной реализации;
	\item реализованы рассматриваемые алгоритмы;
	\item протестировано разработанное ПО;
	\item проведено модульное тестирование всех реализаций алгоритмов;
	\item оценена реализация алгоритмов по времени, памяти и сложности;
	\item оценена актуальность оптимизации алгоритма, приведены краткие рекомендации по особенностям оптимизации.
\end{itemize}

Поставленная цель достигнута.

По итогу выполения лабораторной работы, были реализованы и изучены три алгоритма для умножения матриц: классический алгоритм,
алгоритм Винограда и оптимизированный алгоритм Винограда.
Также были составлены схемы алгоритмов, подсчитана соответствующая трудоёмкость.
Было реализована программа для автоматического замера времени, генерации графиков.
Удалось провести анализ зависимости затрат по времени от размера матрицы..

Рассматривая чётные размеров лучший результат показал оптимизированный алгоритм Винограда, опередив при размере $N=401$
на 2.5\% обычного Винограда. Классический алгоритм сработал на ~18-20\% медленнее.

В случае нечётных размеров наибольшую производительнось опять показал оптимизированный алгоритм Винограда, при размере $N=401$
опередив на 2.5\% обычный алгоритм Винограда и 16\% -- классический алгоритм.

При этом при размере квадратной матрицы $N<6$ классический алгоритм работает быстрее, чем алгоритм Винограда,
и для маленьких матриц такие преобразования не имеют смысл.

Классический алгоритм умножения матриц не подразумевает использование дополнительной памяти, в то время как
алгоритмы Винограда имеют зависимость потребления памяти $O(N)$.
