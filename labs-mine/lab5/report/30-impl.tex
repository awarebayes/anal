\chapter{Технологическая часть}
В данном разделе будут приведены требования к программному обеспечению, средства реализации и листинги кода.

\section{Выбор языка программирования}

В данной лабораторной работе использовался язык программирования - С++ \cite{csharplang}.

Данный язык позволяет создавать нативные потоки, а так же обладает необходимыми примитивыми синхронизации.

В качестве среды разработки выбор сделан в сторону Microsoft Word 2007\cite{wind}. 

Замеры времени производились с помощью функции std::chrono::high\_resolution\_clock::now().

\section{Сведения о модулях программы}
Программа состоит из заголовочного файла queue.h, и файла main.cpp.

\section{Реализация алгоритмов}

За счёт средств языка данный модуль позволяет раздавать задачи из главного потока рабочим потокам. 

Таким образом, в одном модуле описана работа соответствующих потоков, моделирующих ленты конвейера, и все необходимые блокировки: блокировка мьютекса очереди для потокобезопасного добавления и извлечения заявок, а также специфичная для языка т.н. блокировка потока -- средство выдачи задания потоку.


В листинге \ref{lst:con} представлена реализация очереди.

\captionsetup{singlelinecheck = false, justification=raggedright}
\begin{lstlisting}[label=lst:con,caption=Реализация очереди]
template<typename T>
struct Queue
{
private:
    std::queue<T> queue;
    std::mutex mutex;
    std::condition_variable cv;
public:
    void push(T item)
    {
        std::unique_lock<std::mutex> lock(mutex);
        queue.push(item);
        cv.notify_one();
    }

    std::optional<T> pop()
    {
        std::unique_lock<std::mutex> lock(mutex);
        if (cv.wait_for(lock, 10000ms, [this](){ return !queue.empty(); }))
        {
            auto item = queue.front();
            queue.pop();
            return std::move(item);
        }
        return std::nullopt;
    }
};
\end{lstlisting}

В листинге \ref{lst:potok} представлена реализация конвейера.

\begin{lstlisting}[label=lst:potok,caption=Реализация конвейера]
struct Stage
{
    std::string_view stage_name;
    std::function<void()> work;
    int stage_number{};
    std::shared_ptr<Stage> next_stage{};
};

struct StageWithCallback
{
    std::shared_ptr<Stage> stage;
    std::function<void(const Stage &result)> on_complete;
};

struct Task
{
    std::string_view task_name;
    std::vector<std::shared_ptr<Stage>> stages;
    std::chrono::time_point<std::chrono::system_clock> start_time{};
    std::chrono::time_point<std::chrono::system_clock> end_time{};

    Task &add_stage(std::string_view stage_name, std::function<void()> work)
    {
        stages.emplace_back(
			std::make_shared<Stage>(
				Stage{stage_name, work, static_cast<int>(stages.size())}
				)
		);
        if (stages.size() > 1)
        {
            stages[stages.size() - 2]->next_stage = stages[stages.size() - 1];
        }
        return *this;
    }

    Task &complete_building()
    {
        auto stages_0_original_work = stages[0]->work;
        stages[0]->work = [this, stages_0_original_work](){
            this->start_time = std::chrono::system_clock::now();
            stages_0_original_work();
        };

        auto last_stages_original_work = stages[stages.size() - 1]->work;
        stages[stages.size() - 1]->work = [this, last_stages_original_work](){
            last_stages_original_work();
            this->end_time = std::chrono::system_clock::now();
        };
        return *this;
    }
};

struct SequentialPipeline
{
    void run(const std::vector<Task> &tasks)
    {
        for (const auto &task : tasks)
        {
            for (const auto &stage : task.stages)
            {
                stage->work();
            }
        }
    }
};

struct PiplinedPipeline
{
    std::vector<std::shared_ptr<Queue<StageWithCallback>>> queues;
    std::optional<std::reference_wrapper<Queue<
							StageWithCallback>>>
	 			find_queue_with_index(int index)
    {
        if (index < queues.size())
            return *queues[index];
        return std::nullopt;
    }

    void push_result_to_next_queue(const Stage& result)
    {
        auto maybe_next_queue = find_queue_with_index(result.stage_number + 1);
        if (maybe_next_queue)
        {
            auto& next_queue = maybe_next_queue.value().get();
            auto next_stage = result.next_stage;
            next_queue.push({next_stage, [this](const Stage& result_) {
                push_result_to_next_queue(result_);
            }});
        }
    }

    void run(const std::vector<Task> &tasks)
    {
        auto n_stages = tasks[0].stages.size();
        if (queues.size() != n_stages)
        {
            for (int i = 0; i < n_stages; ++i)
            {
                queues.emplace_back(
					std::make_shared<Queue<StageWithCallback>>()
					);
            }
        }

        for (auto& task : tasks)
        {
            auto &stage = task.stages[0];
            auto &entry_queue = queues[0];
            assert(task.stages.size() == n_stages);
            entry_queue->push(StageWithCallback{stage,
            [this](const Stage& result) {
                push_result_to_next_queue(result);
            }});
        }

        std::vector<std::thread> thread_pool;
        int n_tasks = tasks.size();
        for (auto &queue : queues)
        {
            thread_pool.emplace_back([&queue, n_tasks]() {
                int tasks_completed = 0;
                while (tasks_completed < n_tasks)
                {
                    auto maybe_stage = queue->pop();
                    if (maybe_stage)
                    {
                        tasks_completed += 1;
                        auto& stage_with_cb = maybe_stage.value();
                        stage_with_cb.stage->work();
                        stage_with_cb.on_complete(
							*stage_with_cb.stage
							);
                    }
                    else
                        break;
                }
            });
        }

        for (auto& thread: thread_pool)
            thread.join();

    }
};

\end{lstlisting}

В листинге \ref{lst:c1} представлен код задачи, выполняемой на лентах конвейера.

\begin{lstlisting}[label=lst:c1,caption=Код задачи, выполняемой на лентах конвейера]
// guess number by hash.
// number is in range [0, n)
int find_key_by_hash(size_t target_hash, int max_n)
{
	while (true)
	{
		int number = rand() % max_n;
		size_t hash = std::hash<int>{}(number);
		if (hash == target_hash)
		{
			return number;
		}
	}
}

int work(int n)
{
	auto guess = rand() % n;
	size_t target_hash = std::hash<int>{}(guess);
	return find_key_by_hash(target_hash, n);
}
\end{lstlisting}
\captionsetup{singlelinecheck = false, justification=centering}

\section{Тестирование}


Тестирование производилось ручным способом. Работа программы не может завершиться без выполнения 
всех задач. Завершение работы программы говорит о выполнении всех задач.


\section{Вывод}
В данном разделе были разобраны листинги  показывающие работу конвейера. Также был указан способ тестирования.
