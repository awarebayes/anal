\chapter*{Введение}
\addcontentsline{toc}{chapter}{Введение}
Конвейер — способ организации вычислений, используемый в современных процессорах и контроллерах с целью повышения их производительности (увеличения числа инструкций, выполняемых в единицу времени — эксплуатация параллелизма на уровне инструкций), технология, используемая при разработке компьютеров и других цифровых электронных устройств.

Сам термин «конвейер» пришёл из промышленности, где используется подобный принцип работы — материал автоматически подтягивается по ленте конвейера к рабочему, который осуществляет с ним необходимые действия, следующий за ним рабочий выполняет свои функции над получившейся заготовкой, следующий делает ещё что-то и т.д. Таким образом, к концу конвейера цепочка рабочих полностью выполняет все поставленные задачи, сохраняя высокий темп производства. В процессорах роль рабочих исполняют функциональные модули, входящие в состав процессора.


Целью данной лабораторной работы является получение навыков организации асинхронного взаимодействия потоков на примере ковейерной обработки данных.


Для достижения данной цели необходимо решить следующие задачи.


\begin{enumerate}
	\item Изучения основ конвейерной обработки данных.
	\item Применение изученных основ для реализации конвейерной обработки данных.
	\item Получения практических навыков.
	\item Получение статистики выполнения программы.
	\item Выбор и обоснование языка программирования, для решения данной задачи.
	\item Описание и обоснование полученных результатов в отчете о выполненной лабораторной работе, выполненного как расчётно-пояснительная записка к работе.
\end{enumerate}
