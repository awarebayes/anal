\chapter{Технологическая часть}

\section{Выбор языка программирования}

При выполнении домашнего задания использовался язык программирования - С++ \cite{csharplang}.


\section{Исходные код программы}
В листинге \ref{lst:sort} представлена реализация (наивного) алгоритма префиксного суммирования.


\begin{lstlisting}[label=lst:sort,caption=Алгоритм parallel\_scan]

__global__ void prescan(
	float *g_odata,                            // -4
	float *g_idata,                            // -3
	int n 		                               // -2
	)
{
	 extern __shared__ float temp[];           // -1  allocated on invocation
	 // for threadIdx.x = 0 ... n-1 parallel do// 0
	 int thid = threadIdx.x;                   // 1
	 int offset = 1;                           // 2
	 temp[2*thid] = g_idata[2*thid];           // 3 load input into shared memory
	 temp[2*thid+1] = g_idata[2*thid+1];       // 4
	 for (int d = n>>1; d > 0; d >>= 1)        // 5 build sum in place up the tree
	 {
		__syncthreads();                       // 6
		if (thid < d)                          // 7
		{
			int ai = offset*(2*thid+1)-1;      // 8
			int bi = offset*(2*thid+2)-1;      // 9
			temp[bi] += temp[ai];              // 10
		}
		offset *= 2;                           // 11
	 }

	 if (thid == 0) { temp[n - 1] = 0; }       // 12 // clear the last element

	 for (int d = 1; d < n; d *= 2)           // 13 // traverse down tree & build scan
	 {
		offset >>= 1;                         // 14
		__syncthreads();                      // 15
		if (thid < d)                         // 16
		{
			int ai = offset*(2*thid+1)-1;     // 17
			int bi = offset*(2*thid+2)-1;     // 18
			float t = temp[ai];               // 19
			temp[ai] = temp[bi];              // 20
			temp[bi] += t;                    // 21
		}
	 }
	 __syncthreads();                         // 22
	 // write results to device memory
	 g_odata[2*thid] = temp[2*thid];          // 23
	 g_odata[2*thid+1] = temp[2*thid+1];	  // 24
	 // end parallel for
}

\end{lstlisting}

\section{Модели программ}

\subsection{Граф управления программы}

На рисунке \ref{img:og} представлен граф управления программы.

\img{80mm}{og}{Граф управления программы}

\clearpage


\subsection{Информационный граф программы}

На рисунке \ref{img:ig} представлен граф управления программы.

\img{100mm}{ig}{Информационный граф программы}

\clearpage

\subsection{Операционная история программы}

На рисунке \ref{img:oi} представлен граф управления программы.

\img{100mm}{oi}{Операционная история программы}

\clearpage

\subsection{Инофрмационная история программы}

На рисунке \ref{img:ii} представлен граф управления программы.

\img{200mm}{ii}{Информационная история программы}

\clearpage


