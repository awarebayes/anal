\chapter{Аналитическая часть}
В данной лабораторной работе предполагается распараллеливание знакомого нам алгоритма. В качестве такого алгоритма был выбран алгоритм ранговой сортировки. В данной части будет рассмотрено тоеритическое описание алгоритма.

\section{Сортировка}
Сортировка - это процесс перегруппировки заданной последовательности (кортежа) объектов в некотором определеннном порядке. Такой опредленный порядок позволяет, в некоторых случаях, эффективнее и удобнее
работать с заданной последовательностью. В частности, одной из целей
сортировки является облегчение задачи поиска элемента в отсортированном множестве.


Существует множество различных методов сортировки данных. Однако любой алгоритм сортировки можно разбить на три основные части:

\begin{itemize}
	\item сравнение, определяющее упорядочность пары элементов;
	\item перестановка, меняющая местами пару элементов;
	\item собственно сортирующий алгоритм, который осуществляет сравнение
	и перестановку элементов данных до тех пор, пока все эти элементы
	не будут упорядочены.
\end{itemize}

Одной из важнейшей характеристикой любого алгоритма сортировки является скорость его работы, которая определяется функциональной зависимостью среднего времени сортировки последовательностей элементов данных, определенной длины, от этой длины.


\section{Ранговая сортировка последовательности}

Aлгоритмы ранговой сортировки \cite{sortr} основаны на вычислении рангов элементов и их
упорядочении в соответствии с полученным рангом. Ранги элементов могут быть определены любым способом, однако в практике зачастую достаточным является следующая
формула вычисления ранга:
 
 \begin{equation}
 	\label{for:selection_best}
 	rank(s_{i}) = \sum rank(R(s_{i}, s_{j})),
 \end{equation}

где ранжируются результаты $(R(s_{i}, s_{j})$ операций попарно сравнения элементов и выполняется суммирование полученных рангов.

Алгоритм можно выполнить следующими шагами для каждого элемента массива.
\begin{enumerate}
	\item Подсчет количества чисел, меньше, чем исследуемое (в случае, если числа равны, то сравнивается индекс).
	\item В результирующий массив в ячейку с индексом, равным количеству числе, меньших, чем исследуемое (подсчитано в шаге 1) поставить исследуемое число.
	\item Повторить шаги 1-2 для всех элементов массива.
\end{enumerate}

Пример по шагам представлен на рисунке \ref{img:prim}.

\clearpage

\section*{Вывод}
В данном разделе были рассмотрены основополагающие материалы, которые в дальнейшем потребуются при параллельной и однопоточной реализации алгоритма ранговой сортировки.

