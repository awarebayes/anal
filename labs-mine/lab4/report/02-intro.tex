\chapter*{\hfill{\centering Введение }\hfill}
\addcontentsline{toc}{chapter}{Введение}

Параллельные вычисления -- способ организации компьютерных вычислений, при котором программы разрабатываются как набор взаимодействующих вычислительных процессов, работающих параллельно (одновременно).
Параллельные вычисления позволяют существенно ускорить некоторые вычисления.

В работе внимание будет уделено реализации программы на графическом ускорителе.
Мы не будем использовать термин многопоточность, так как он не применителен графическим ускорителям, заменив его параллельностью.
Парадигма параллельности обрела популярность конце 1990-х готов, поскольку один процессор не мог обрабатывать все задачи.

GPGPU -- это техника использования графического процессора видеокарты, предназначенного для компьютерной графики, в целях производства математических вычислений, которые обычно проводит центральный процессор. Это стало возможным благодаря добавлению программируемых шейдерных блоков и более высокой арифметической точности растровых конвейеров, что позволяет разработчикам ПО использовать потоковые процессоры видеокарт для выполнения неграфических вычислений. [cite wiki]
Архитектура графического ускорителя существенно отличается от процессора, чему будет уделено дополнительное внимание.

В этой работе будет рассматриваться реализация параллельного программного растеризатора на GPGPU.

Задачи данной лабораторной:

\begin{itemize}
	\item изучить основы параллельных вычислений;
	\item проанализировать существующую реализацию курсовой работы;
	\item получить практические навыки;
	\item провести сравнительный анализ реализаций с одним потоком и со многими;
	\item обосновать выбор языка программирования;
	\item описать и обосновать полученные результаты в отчете.
\end{itemize}
