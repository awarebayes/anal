\chapter{Технологическая часть}

В данном разделе будут приведены требования к программному обеспечению, средства реализации и листинги кода.

\section{Требования к ПО}

К программе предъявляется ряд требований:
\begin{itemize}
	\item у пользователя есть выбор алгоритма, или какой-то один, или все сразу, а также есть выбор тестирования времени;
	\item на вход подаются две строки на русском или английском языке в любом регистре;
	\item на выходе — искомое расстояние для выбранного метода (выбранных методов) и матрицы расстояний для матричных реализаций.
\end{itemize}

\section{Средства реализации}

В качестве языка программирования для реализации данной лабораторной работы был выбран язык программирования Python \cite{pythonlang}. 

Данный язык прост в использовании, предлагает большой набор библиотек для работы с таблицами и графиками,
необходимых для составления этого отчета.

Время работы алгоритмов было замерено с помощью функции \newline process\_time() из библиотеки time \cite{pythonlangtime}.

\section{Сведения о модулях программы}
Программа состоит из двух модулей:
\begin{itemize}
	\item main.py - главный файл программы, в котором располагаются коды всех алгоритмов и меню;
	\item test.py - файл с замерами времени для графического изображения результата.
\end{itemize}


\section{Реализация алгоритмов}

 В листингах \ref{lst:lev_rec}, \ref{lst:dlev}, \ref{lst:dlev_matrix}, \ref{lst:dlev_matrix_rec} приведены реализации алгоритмов нахождения расстояния Левенштейна и Дамерау-Левенштейна.

\begin{lstlisting}[label=lst:lev_rec,caption=Функция нахождения расстояния Левенштейна с использованием рекурсии.]
def levenshtein_recursive(str1, str2):
    n = len(str1)
    m = len(str2)

    if n == 0 or m == 0:
        return abs(n - m)

    flag = int(str1[-1] != str2[-1])

    min_lev_rec = min(levenshtein_recursive(str1[:-1], str2, profiler) + 1,
                      levenshtein_recursive(str1, str2[:-1], profiler) + 1,
                      levenshtein_recursive(str1[:-1], str2[:-1], profiler) + flag)
	return min_lev_rec
	
\end{lstlisting}

\begin{lstlisting}[label=lst:dlev,caption=Функция нахождения расстояния Дамерау-Левенштейна с использованием рекурсии.]
def damerau_levenshtein_recursive(str1, str2, out_put = False):
	n = len(str1)
	m = len(str2)
	
	if n == 0 or m == 0:
		return abs(n - m)
	
	flag = str1[-1] != str2[-1]:
	
	min_ret = min(
		damerau_levenshtein_recursive(str1[:n - 1], str2) + 1,
		damerau_levenshtein_recursive(str1, str2[:m - 1]) + 1,
		damerau_levenshtein_recursive(str1[:n - 1], str2[:m - 1]) + change
	)
	if n > 1 and m > 1 and str1[-1] == str2[-2] \
		and str1[-2] == str2[-1]:
		min_ret = min(min_ret,
		              damerau_levenshtein_recursive(str1[:n - 1], str2[:m - 1]) + change)
	return min_ret
	
\end{lstlisting}

\begin{lstlisting}[label=lst:dlev_matrix,caption=Функция нахождения расстояния Дамерау Левенштейна с использованием кеша.]
def damerau_levenshtein_matrix(str1, str2, out_put = True):
     n = len(str1)
     m = len(str2)

     matrix = create_matrix(n + 1, m + 1)
     
     for i in range(1, n + 1):
         for j in range(1, m + 1):
             add, delete, change = matrix[i - 1][j] + 1, \
                                   matrix[i][j - 1] + 1, \
                                   matrix[i - 1][j - 1]
             if str2[j - 1] != str1[i - 1]:
                 change += 1

             matrix[i][j] = min(add, delete, change)
             if i > 1 and j > 1 and str1[i - 1] == str2[j - 2] \
                     and str1[i - 2] == str2[j - 1]:
                 matrix[i][j] = min(matrix[i][j], matrix[i - 2][j - 2] + 1)
     return matrix[n][m]
\end{lstlisting}

\begin{lstlisting}[label=lst:dlev_matrix_rec,caption=Функция нахождения расстояния Дамерау Левенштейна с использованием кеша и рекурсии.]
def damerau_levenshtein_cached(str1, str2):
    n = len(str1)
    m = len(str2)

    def recursive(str1, str2, n, m, matrix):
        profiler.trace()
        if (matrix[n][m] != -1):
            return matrix[n][m]

        if (n == 0):
            matrix[n][m] = m
            return matrix[n][m]

        if (n > 0 and m == 0):
            matrix[n][m] = n
            return matrix[n][m]

        delete = recursive(str1, str2, n - 1, m, matrix) + 1
        add = recursive(str1, str2, n, m - 1, matrix) + 1

        flag = str1[n - 1] != str2[m - 1]
        change = recursive(str1, str2, n - 1, m - 1, matrix) + flag

        matrix[n][m] = min(add, delete, change)
        if n > 1 and m > 1 and str1[n - 1] == str2[m - 2] \
                     and str1[n - 2] == str2[m - 1]:
                 matrix[n][m] = min(matrix[n][m], matrix[n - 2][m - 2] + 1)

        return matrix[n][m]

    matrix = create_matrix(n + 1, m + 1)
    for i in range(n + 1):
        for j in range(m + 1):
            matrix[i][j] = -1

    recursive(str1, str2, n, m, matrix)
    return matrix[n][m]

\end{lstlisting}

\section{Функциональные тесты}
В таблице \ref{tabular:functional_test} приведены функциональные тесты для алгоритмов вычисления расстояния Левенштейна (в таблице столбец подписан "Левенштейн") и Дамерау — Левенштейна (в таблице - "Дамерау-Л."). Все тесты пройдены успешно.
Тестирование проводилось методом черного ящика.


\begin{table}[h!]
	\centering
	\begin{center}
		\begin{threeparttable}
		\caption{\label{tabular:functional_test} Функциональные тесты}
		\begin{tabular}{|c|c|c|c|c|}
			\hline
			& & & \multicolumn{2}{c|}{Ожидаемый результат} \\
			\hline
			№&Строка 1&Строка 2&Левенштейн&Дамерау-Л. \\
			\hline
			1&скат&кот&2&2 \\
			\hline
			2&машина&малина&1&1 \\
			\hline
			3&дворик&доврик&2&1 \\
			\hline
			4&"пустая строка"&университет&11&11 \\
			\hline
			5&сентябрь&"пустая строка"&8&8 \\
			\hline
			8&тело&телодвижение&8&8 \\
			\hline
			9&ноутбук&планшет&7&7 \\
			\hline
			10&глина&малина&2&2 \\
			\hline
			11&рекурсия&ркерусия&3&2 \\
			\hline
			12&браузер&баурзер&2&2 \\
			\hline
			13&bring&brought&4&4 \\
			\hline
			14&moment&minute&4&4 \\ 
			\hline
			15&person&eye&5&5 \\
			\hline
			16&week&weekend&3&3 \\
			\hline 
			17&city&town&4&4 \\
			\hline
		\end{tabular}
		\end{threeparttable}
	\end{center}
\end{table}


\section*{Вывод}

Были реализованы и протестированы алгоритмы: нахождения расстояния Левенштейна рекурсивно, с заполнением матрицы и рекурсивно с заполнением матрицы, а также нахождения расстояния Дамерау — Левенштейна рекурсивно, Дамерау - Левенштейна с заполнением матрицы, Дамерау - Левенштейна с заполнением матрицы и рекурсией.
