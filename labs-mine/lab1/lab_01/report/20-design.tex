\chapter{Конструкторская часть}
В этом разделе будут приведены требования к вводу и программе, а также схемы алгоритмов нахождения расстояний Левенштейна и Дамерау-Левенштейна.

\section{Требования к вводу}
\begin{enumerate}[label={\arabic*)}]
	\item на вход подаются две строки;
	\item буквы верхнего и нижнего регистров считаются различными.
\end{enumerate}

\section{Требования к программе}
\begin{enumerate}[label={\arabic*)}]
	\item две пустые строки - корректный ввод, программа не должна аварийно завершаться;
	\item на выход программа должна вывести число - расстояние Левенштейна (Дамерау-Левенштейна), матрицу при необходимости.
\end{enumerate}

\section{Разработка алгоритма нахождения расстояния Левенштейна}

На рисунке \ref{img:diagrams_levenstein_rec} приведена схема рекурсивного алгоритма нахождения расстояния Левенштейна.

\section{Разработка алгоритма нахождения расстояния Дамерау - Левенштейна}

На рисунке \ref{img:diagrams_dam_lev_matrix} приведена Разработка алгоритма нахождения расстояния Дамерау -- Левенштейна с заполнением матрицы.
На рисунке \ref{img:diagrams_dam_lev_recursive} приведена схема рекурсивного алгоритма нахождения расстояния Дамерау -- Левенштейна.
На рисунке \ref{img:diagrams_damerau_levenstein_recursive_matrix} и \ref{img:diagrams_damerau_levenstein_recursive} приведена схема рекурсивного алгоритма нахождения расстояния Дамерау -- Левенштейна c использованием кеша (заполнением матрицы).

\img{180mm}{diagrams_levenstein_rec}{Схема рекурсивного алгоритма нахождения расстояния Левенштейна}
\img{220mm}{diagrams_dam_lev_matrix}{Схема матричного алгоритма нахождения расстояния Дамерау-Левенштейна}
\img{220mm}{diagrams_dam_lev_recursive}{Схема рекурсивного алгоритма нахождения расстояния Дамерау-Левенштейна}
\img{220mm}{diagrams_damerau_levenstein_recursive_matrix}{Схема рекурсивного матричного алгоритма нахождения расстояния Дамерау-Левенштейна (1)}
\img{220mm}{diagrams_damerau_levenstein_recursive}{Схема рекурсивного матричного алгоритма нахождения расстояния Дамерау-Левенштейна (2)}

\section*{Вывод}

Перечислены требования к вводу и программе, а также на основе теоретических данных, полученных из аналитического раздела были построены схемы требуемых алгоритмов.